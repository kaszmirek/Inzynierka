\chapter{Preliminaria matematyczne}\label{ch_02}
Celem rozdziału jest zaprezentowanie podstawowych informacji dotyczących aparatu matematycznego wykorzystywanego w pracy. Ma on ponadto ułatwić dalsze czytanie poprzez zaznajomienie czytelnika z przyjętymi konwencjami oznaczeń oraz symboli, które można napotkać w następnych rozdziałach. Przedstawione zostaną tu również interpretacje wprowadzonych wzorów oraz formalizmów. Szczegółowe opisy oraz dowody przedstawianych zagadnień oraz twierdzeń można znaleźć w~większości dostępnych podręczników do mechaniki analitycznej. Niniejsza praca bazuje w głównej mierze na pozycjach \cite{TchMu18, RubKro12, Arn81,LegMakeSens}. 

{\red
  Każdy rozdział warto rozpocząć od podania informacji, w~jakim celu został on napisany i~co zawiera. We wstępie do rozdziału mogą też pojawić się inne informacje natury ogólnej: rys historyczny, metodologia postępowania, charakterystyka zastosowanych narzędzi, własności proponowanego rozwiązania.}

\section{Ruch bryły sztywnej}\label{ch_02:ruch}
W ogólności ruch bryły sztywnej jest złożeniem dwóch podstawowych ruchów -- przesunięcia oraz obrotu ciała sztywnego. Opis takiego ruchu bazujący na trójelementowym wektorze położenia $(x,y,z)^T$ z elementami $x$, $y$, $z$ jednoznacznie określającymi położenie w przestrzeni euklidesowej $\mathbb{E}^3$. [\ldots] Elementy grupy $\mathbb{SE}(3)$ można utożsamić z macierzami 4x4, które przyjmują postać
\begin{equation} \label{equ:transf}
  \boldsymbol{A} = \begin{bmatrix}
    \boldsymbol{R} & \boldsymbol{T} \\
    \boldsymbol{0}^T & 1 
  \end{bmatrix},
\end{equation}
gdzie $\RR$ -- macierz 3$\times$3 opisująca część rotacyjną ruchu, $\TT$ -- trójelementowy wektor opisujący część translacyjną ruchu.  

{\red
  Wzory wystawione stanowią integralną część zawierających je zdań. Należy więc konstruować całość zgodnie z~zasadami gramatyki języka polskiego, stosując odpowiednią interpunkcję (przecinek/kropka po równaniu?). Po podaniu równania należy określić znaczenie wszystkich użytych w~nim elementów, które wcześniej nie były zdefiniowane (dotyczy też formuł umieszczanych w~tekście -- zobacz wyżej definicję wektora położenia). Wzory wystawione można uzyskać używając otoczeń \texttt{equation}, \texttt{equation*}\footnote{\red Z~numerem, bez numeru -- zasadniczo numerujemy tylko wzory, do których się odwołujemy, czasem te uważane za ,,ważne''. By numery automatycznie pojawiały się jedynie przy wzorach do których są w~tekście odwołania można dołączyć pakiet \texttt{mathtools}, w~tekście dokumentu użyć polecenia \texttt{\textbackslash mathtoolsset\{showonlyrefs=true\}}, a~następnie używać otoczenia \texttt{equation} w~wersji bez gwiazdki. Pakiet \texttt{mathtools} daje oczywiście dużo więcej możliwości \cite{mathtools}.}\footnote{\red  Jeśli równanie stanowi jeden akapit z~otaczającym je tekstem, przed/po nim nie może pojawić się w~pliku źródłowym pusta linia.}. Macierze, wektory konstruujemy korzystając z~otoczeń \texttt{bmatrix}, \texttt{pmatrix}. Więcej o~równaniach na początku rozdziału~\ref{czym_jest_bak}.}

\noindent
[\ldots] Ilustracja przekształcenia została przedstawiona na rysunku \ref{fig:transf_se3}.
\begin{figure} [tp]
  \centering%
  \scalebox{1.0}{\def\svgscale{0.67} %\def\svgwidth{0.4\textwidth} %%alternatywnie do wyboru
  
\definecolor{c1414c8}{RGB}{20,20,200}
\definecolor{c5e81b5}{RGB}{94,129,181}


\begin{tikzpicture}[y=0.80pt, x=0.8pt,yscale=-1, inner sep=0pt, outer sep=0pt]
\begin{scope}[shift={(0,-308.2677)}]
  \path[color=black,draw=black,line join=miter,line cap=butt,miter limit=4.00,even
    odd rule,line width=0.600pt] (594.5408,643.1321) -- (296.4378,643.1321);
  \path[draw=black,line join=miter,line cap=butt,miter limit=4.00,even odd
    rule,line width=0.600pt] (333.1481,449.6988) -- (333.1481,676.1777);
  \path[color=black,draw=black,line join=miter,line cap=butt,miter limit=4.00,even
    odd rule,line width=0.600pt] (276.0863,699.8907) -- (351.6971,624.5647);
  \path[color=black,draw=c1414c8,dash pattern=on 4.50pt off 2.25pt,line
    join=miter,line cap=butt,miter limit=4.00,nonzero rule,line width=0.562pt]
    (541.6406,475.4327) -- (588.3634,482.9315);
  \path[color=black,draw=c1414c8,dash pattern=on 4.50pt off 2.25pt,line
    join=miter,line cap=butt,miter limit=4.00,even odd rule,line width=0.562pt]
    (614.9085,506.4173) -- (599.3615,534.9176);
  \path[color=black,draw=black,line join=miter,line cap=butt,miter limit=4.00,even
    odd rule,line width=0.600pt] (647.0396,497.2400) -- (535.5269,528.9546);
  \path[color=black,draw=c1414c8,dash pattern=on 4.50pt off 2.25pt,line
    join=miter,line cap=butt,miter limit=4.00,even odd rule,line width=0.562pt]
    (541.7165,547.8752) -- (606.6698,529.5082);
  \path[draw=black,line join=miter,line cap=butt,miter limit=4.00,even odd
    rule,line width=0.600pt] (531.0722,438.9698) -- (559.1724,537.7867);
  \path[color=black,draw=black,line join=miter,line cap=butt,miter limit=4.00,even
    odd rule,line width=0.600pt] (529.8444,569.5425) -- (559.7774,515.0070);
  \path[draw=c1414c8,line join=miter,line cap=butt,miter limit=4.00,line
    width=0.600pt] (585.6805,485.1590) -- (555.0695,523.3737);
  \path[draw=c1414c8,dash pattern=on 4.50pt off 2.25pt,line join=miter,line
    cap=butt,miter limit=4.00,even odd rule,dash phase=3.431pt,line width=0.562pt]
    (588.3511,484.0313) -- (602.5613,534.4376);
  \path[color=black,draw=c5e81b5,fill=c5e81b5,line join=miter,line cap=round,miter
    limit=4.00,nonzero rule,line width=0.600pt]
    (589.7119,482.5042)arc(-8.940:80.993:2.115845 and
    2.116)arc(80.993:170.926:2.115845 and 2.116)arc(170.926:260.860:2.115845 and
    2.116)arc(260.860:350.793:2.115845 and 2.116);
  \path[color=black,draw=c1414c8,dash pattern=on 4.50pt off 2.25pt,line
    join=miter,line cap=butt,miter limit=4.00,nonzero rule,line width=0.562pt]
    (555.0695,523.3737) -- (601.7923,530.8725);
  \path[xscale=1.000,yscale=1.000,fill=black,line join=miter,line cap=butt,line
    width=0.800pt] (583.31714,666.13422) node[above right] (text8673) {$Y_S$};
  \path[xscale=1.000,yscale=1.000,fill=black,line join=miter,line cap=butt,line
    width=0.800pt] (342.8504,460.09625) node[above right] (text8677) {$Z_S$};
  \path[xscale=1.000,yscale=1.000,fill=black,line join=miter,line cap=butt,line
    width=0.800pt] (590.75604,478.96622) node[above right] (text10221) {$P$};
  \path[xscale=1.000,yscale=1.000,fill=black,line join=miter,line cap=butt,line
    width=0.800pt] (285.86298,714.60553) node[above right] (text6526) {$X_S$};
  \path[xscale=1.000,yscale=1.000,fill=black,line join=miter,line cap=butt,line
    width=0.800pt] (464.52386,438.00558) node[above right] (text6597) {$B$};
  \path[xscale=1.000,yscale=1.000,fill=black,line join=miter,line cap=butt,line
    width=0.800pt] (537.28204,445.66144) node[above right] (text4638) {$Z_B$};
  \path[xscale=1.000,yscale=1.000,fill=black,line join=miter,line cap=butt,line
    width=0.800pt] (638.49066,519.0498) node[above right] (text4642) {$Y_B$};
  \path[xscale=1.000,yscale=1.000,fill=black,line join=miter,line cap=butt,line
    width=0.800pt] (536.8288,578.54279) node[above right] (text4646) {$X_B$};
  \path[xscale=1.000,yscale=1.000,fill=black,line join=miter,line cap=butt,line
    width=0.800pt] (562.47626,499.48889) node[above right] (text9904) {$r$};
  \path[color=black,draw=c5e81b5,line join=miter,line cap=butt,miter
    limit=4.00,nonzero rule,line width=1.875pt] (454.0379,467.8855) .. controls
    (454.2694,450.2157) and (453.4335,424.9121) .. (467.9923,416.7412) .. controls
    (492.4315,403.0250) and (519.8902,410.1543) .. (549.1789,417.4173) .. controls
    (568.7201,422.2631) and (590.6036,423.6038) .. (607.3340,431.9532) .. controls
    (631.4798,444.0033) and (659.5551,461.6455) .. (674.9471,484.3528) .. controls
    (681.5303,494.0649) and (676.6154,509.9043) .. (677.2583,521.4283) .. controls
    (678.0787,536.1310) and (685.1670,561.0955) .. (679.7756,575.4243) .. controls
    (666.9437,609.5278) and (629.0839,607.6335) .. (596.9898,614.3785) .. controls
    (565.3454,621.0290) and (545.5650,623.3195) .. (519.5088,598.0712) .. controls
    (508.0193,586.9380) and (487.4233,576.5857) .. (477.4237,563.3846) .. controls
    (457.2463,534.4323) and (453.6084,500.6563) .. (454.0379,467.8855) -- cycle;
  \path[draw=c1414c8,line join=miter,line cap=butt,miter limit=4.00,line
    width=0.600pt] (552.5412,522.5818) .. controls (414.4414,497.0423) and
    (359.6495,574.1987) .. (334.6637,639.6009);
  \path[rotate=-34.12828,fill=black,line join=miter,line cap=butt,line
    width=0.750pt] (-31.111509,718.05353) node[above right] (text3922) {\small
    A=$\begin{bmatrix}R&T\\0^T&1\end{bmatrix}$};
\end{scope}

\end{tikzpicture}

}
  \caption{Przekształcenie układów współrzędnych, \cite{TchMu18}}
  \label{fig:transf_se3}
\end{figure}
%%
%% alternatywna wersja z~użyciem includesvg, który automatycznie konwertuje svg to pdf
%% jeśli plik svg jest nowszy - niestety nie zawsze działa i w starszych wersjach svg daje zły boundingbox (page)
%%
% \begin{figure} [tp]
%   \centering
% %%  \includesvg[scale=0.67]{figures/chapter_02/przemieszczenie}
% %%                                          w~starszych wersjach pakietu svg nie ma opcji scale :(
%   \includesvg[width=0.5\textwidth]{figures/chapter_02/przemieszczenie}
%   \caption{Przekształcenie układów współrzędnych, \cite{TchMu18}}
%   \label{fig:transf_se3}
% \end{figure}
Ze względu na charakter ruchu badanych obiektów, w pracy zajmiemy się tylko częścią rotacyjną przekształcenia (\ref{equ:transf}), która pozwoli nam na analizę ewolucji zachowań bąków.

{\red
  Rysunki wektorowe dobrze jest opracowywać z~wykorzystaniem programu Inkscape\footnote{\red Który potrafi je zapisać w~formacie \texttt{PDF} z~wydzieloną warstwą tekstową. W~tym celu zapisujemy plik jako dokument w~formacie \texttt{PDF} i~w~pojawiającym się oknie konfiguracji zaznaczamy "Omit text in PDF and create LaTeX file" oraz "Użyj rozmiar strony eksportowanego obiektu" -- uzyskujemy w~ten sposób dwa pliki, odpowiednio z~rozszerzeniami \texttt{.pdf} i \texttt{.pdf\_tex}\footnotemark, z~których drugi dołączamy do pliku głównego poleceniem \texttt{\textbackslash input\{\}}.  W~załączeniu znajduje się plik źródłowy Inkscape'a \texttt{przemieszczenie.svg} (w katalogu \texttt{figures/chapter\_02}), zawierający grafikę z~rysunku~\ref{fig:transf_se3}.}\footnotetext{\red Konwersji źródłowego pliku zapisanego w~formacie \texttt{svg} można dokonać też z~poziomu shella poleceniem \texttt{inkscape -z -D --file=<infile.svg> --export-pdf=<outfile.pdf> --export-latex}. Alternatywnie, użycie w~miejsce polecenia \texttt{\textbackslash input\{\}} polecenia \texttt{\textbackslash includesvg\{\}} z~pakietu \texttt{svg} powoduje automatyczne przetwarzanie pliku \texttt{svg} w~\texttt{pdf} przy każdym uruchomieniu kompilatora pdflatecha\footnotemark, jeśli tylko plik \texttt{pdf} nie jest aktualny -- niestety nie we wszystkich systemach operacyjnych działa to poprawnie. W~źródłach tego dokumentu zaprezentowano oba sposoby.}\footnotetext{\red wywołanego być może z~dodatkową opcją \texttt{--shell-escape}}  \cite{inkscape} lub podobnego. Pozwala to na przygotowanie grafik wektorowych jak ta z~rysunku~\ref{fig:transf_se3} z~czcionkami spójnymi z~tymi w~tekście głównym. Wielkość rysunku można kontrolować przez podanie wartości parametru \texttt{\textbackslash svgscale} (skaluje elementy graficzne pozostawiając wielkość i~położenie czcionek bez zmian!) lub użycie polecenia \texttt{\textbackslash scalebox}\footnote{\red W~pliku źródłówym użyto przykładowo tego polecenia ze skalą 1.0.}. Alternatywnie rysunek z~Inkscape'a można dołączyć zapisując go w~formacie Ti{\it k}Z, ale to już zupełnie inna bajka.}
%% \begin{figure} [tp]
%%   \centering%
%%   \scalebox{1.0}{\def\svgscale{0.67}
%%   
\definecolor{c1414c8}{RGB}{20,20,200}
\definecolor{c5e81b5}{RGB}{94,129,181}


\begin{tikzpicture}[y=0.80pt, x=0.8pt,yscale=-1, inner sep=0pt, outer sep=0pt]
\begin{scope}[shift={(0,-308.2677)}]
  \path[color=black,draw=black,line join=miter,line cap=butt,miter limit=4.00,even
    odd rule,line width=0.600pt] (594.5408,643.1321) -- (296.4378,643.1321);
  \path[draw=black,line join=miter,line cap=butt,miter limit=4.00,even odd
    rule,line width=0.600pt] (333.1481,449.6988) -- (333.1481,676.1777);
  \path[color=black,draw=black,line join=miter,line cap=butt,miter limit=4.00,even
    odd rule,line width=0.600pt] (276.0863,699.8907) -- (351.6971,624.5647);
  \path[color=black,draw=c1414c8,dash pattern=on 4.50pt off 2.25pt,line
    join=miter,line cap=butt,miter limit=4.00,nonzero rule,line width=0.562pt]
    (541.6406,475.4327) -- (588.3634,482.9315);
  \path[color=black,draw=c1414c8,dash pattern=on 4.50pt off 2.25pt,line
    join=miter,line cap=butt,miter limit=4.00,even odd rule,line width=0.562pt]
    (614.9085,506.4173) -- (599.3615,534.9176);
  \path[color=black,draw=black,line join=miter,line cap=butt,miter limit=4.00,even
    odd rule,line width=0.600pt] (647.0396,497.2400) -- (535.5269,528.9546);
  \path[color=black,draw=c1414c8,dash pattern=on 4.50pt off 2.25pt,line
    join=miter,line cap=butt,miter limit=4.00,even odd rule,line width=0.562pt]
    (541.7165,547.8752) -- (606.6698,529.5082);
  \path[draw=black,line join=miter,line cap=butt,miter limit=4.00,even odd
    rule,line width=0.600pt] (531.0722,438.9698) -- (559.1724,537.7867);
  \path[color=black,draw=black,line join=miter,line cap=butt,miter limit=4.00,even
    odd rule,line width=0.600pt] (529.8444,569.5425) -- (559.7774,515.0070);
  \path[draw=c1414c8,line join=miter,line cap=butt,miter limit=4.00,line
    width=0.600pt] (585.6805,485.1590) -- (555.0695,523.3737);
  \path[draw=c1414c8,dash pattern=on 4.50pt off 2.25pt,line join=miter,line
    cap=butt,miter limit=4.00,even odd rule,dash phase=3.431pt,line width=0.562pt]
    (588.3511,484.0313) -- (602.5613,534.4376);
  \path[color=black,draw=c5e81b5,fill=c5e81b5,line join=miter,line cap=round,miter
    limit=4.00,nonzero rule,line width=0.600pt]
    (589.7119,482.5042)arc(-8.940:80.993:2.115845 and
    2.116)arc(80.993:170.926:2.115845 and 2.116)arc(170.926:260.860:2.115845 and
    2.116)arc(260.860:350.793:2.115845 and 2.116);
  \path[color=black,draw=c1414c8,dash pattern=on 4.50pt off 2.25pt,line
    join=miter,line cap=butt,miter limit=4.00,nonzero rule,line width=0.562pt]
    (555.0695,523.3737) -- (601.7923,530.8725);
  \path[xscale=1.000,yscale=1.000,fill=black,line join=miter,line cap=butt,line
    width=0.800pt] (583.31714,666.13422) node[above right] (text8673) {$Y_S$};
  \path[xscale=1.000,yscale=1.000,fill=black,line join=miter,line cap=butt,line
    width=0.800pt] (342.8504,460.09625) node[above right] (text8677) {$Z_S$};
  \path[xscale=1.000,yscale=1.000,fill=black,line join=miter,line cap=butt,line
    width=0.800pt] (590.75604,478.96622) node[above right] (text10221) {$P$};
  \path[xscale=1.000,yscale=1.000,fill=black,line join=miter,line cap=butt,line
    width=0.800pt] (285.86298,714.60553) node[above right] (text6526) {$X_S$};
  \path[xscale=1.000,yscale=1.000,fill=black,line join=miter,line cap=butt,line
    width=0.800pt] (464.52386,438.00558) node[above right] (text6597) {$B$};
  \path[xscale=1.000,yscale=1.000,fill=black,line join=miter,line cap=butt,line
    width=0.800pt] (537.28204,445.66144) node[above right] (text4638) {$Z_B$};
  \path[xscale=1.000,yscale=1.000,fill=black,line join=miter,line cap=butt,line
    width=0.800pt] (638.49066,519.0498) node[above right] (text4642) {$Y_B$};
  \path[xscale=1.000,yscale=1.000,fill=black,line join=miter,line cap=butt,line
    width=0.800pt] (536.8288,578.54279) node[above right] (text4646) {$X_B$};
  \path[xscale=1.000,yscale=1.000,fill=black,line join=miter,line cap=butt,line
    width=0.800pt] (562.47626,499.48889) node[above right] (text9904) {$r$};
  \path[color=black,draw=c5e81b5,line join=miter,line cap=butt,miter
    limit=4.00,nonzero rule,line width=1.875pt] (454.0379,467.8855) .. controls
    (454.2694,450.2157) and (453.4335,424.9121) .. (467.9923,416.7412) .. controls
    (492.4315,403.0250) and (519.8902,410.1543) .. (549.1789,417.4173) .. controls
    (568.7201,422.2631) and (590.6036,423.6038) .. (607.3340,431.9532) .. controls
    (631.4798,444.0033) and (659.5551,461.6455) .. (674.9471,484.3528) .. controls
    (681.5303,494.0649) and (676.6154,509.9043) .. (677.2583,521.4283) .. controls
    (678.0787,536.1310) and (685.1670,561.0955) .. (679.7756,575.4243) .. controls
    (666.9437,609.5278) and (629.0839,607.6335) .. (596.9898,614.3785) .. controls
    (565.3454,621.0290) and (545.5650,623.3195) .. (519.5088,598.0712) .. controls
    (508.0193,586.9380) and (487.4233,576.5857) .. (477.4237,563.3846) .. controls
    (457.2463,534.4323) and (453.6084,500.6563) .. (454.0379,467.8855) -- cycle;
  \path[draw=c1414c8,line join=miter,line cap=butt,miter limit=4.00,line
    width=0.600pt] (552.5412,522.5818) .. controls (414.4414,497.0423) and
    (359.6495,574.1987) .. (334.6637,639.6009);
  \path[rotate=-34.12828,fill=black,line join=miter,line cap=butt,line
    width=0.750pt] (-31.111509,718.05353) node[above right] (text3922) {\small
    A=$\begin{bmatrix}R&T\\0^T&1\end{bmatrix}$};
\end{scope}

\end{tikzpicture}

}
%%   \caption{Przekształcenie układów współrzędnych (eksport przez Ti{\it k}Za), \cite{TchMu18}}
%%   \label{fig:transf_se3_tikz}
%% \end{figure}



\noindent
[\ldots]

\section{Formalizm Eulera-Newtona} \label{sec:newton}
\noindent
[\ldots] Układ równań Eulera-Newtona opisujący dynamikę bryły przyjmuje wtedy postać
\begin{equation}
  \label{rEN}
  \begin{cases}
    \FF = m_b \left(\dot{\pmb{v}}_B + \pmb{\omega}_B\times\dot{\pmb{v}}_B \right)\\
    \angmom = I_B\pmb{\dot{\omega}}_B + \pmb{\omega}_B\times I_B\pmb{\omega}_B
  \end{cases},
\end{equation}
gdzie $\FF$ -- siła zewnętrzna działająca na\ldots

{\red
  Układy równań wygodnie jest robić z~wykorzystaniem otoczenia \texttt{cases}, jak to zrobiono w~równaniu~\eqref{rEN}.}

\noindent
[\ldots]

\noindent
Dla tak zdefiniowanego lagranżianu równania ruchu wyrażone są za pomocą równań Eulera-Lagrange'a drugiego rodzaju postaci
\begin{equation*}
  \frac{d}{dt}\left( \frac{\partial(L(\qq,\dot{\qq})}{\partial\dot{\qq}}\right) - \frac{\partial L(\qq,\dot{\qq})}{\partial \qq} = \FF,
\end{equation*}
gdzie $\FF$ oznacza siły\ldots

{\red
  Jeśli dane równanie ma nie być opatrzone numerem używamy ,,starowanych'' wersji otoczeń matematycznych (np. \texttt{equation*} w~miejsce \texttt{equation})\footnote{\label{stopka_numery}\red W~kwestii numerowania równań matematycznych istnieją dwie szkoły. Według pierwszej numerujemy ,,ważniejsze'' równania (w~tym oczywiście te, do których są odwołania w~tekście), te ,,mniej ważne'' pozostawiając bez numerów. Druga szkoła mówi, że należy numerować jedynie te równania, do których są odwołania w~tekście. I~tutaj w~sukurs przychodzi nam pakiet \texttt{mathtools} i~jego opcja \texttt{showonlyrefs} -- wówczas wszystkie równania możemy dawać w~wersji bez gwiazdek na końcu, a~numery otrzymają jedynie te równania, do których pojawią się odwołania skonstruowane za pomocą mechanizmu odsyłaczy \texttt{\textbackslash label\{\}}-\texttt{\textbackslash eqref\{\}} (odkomentuj w~pliku \texttt{main.tex} odpowiednie linie, by zobaczyć efekt).}. W~celu dostosowania wielkości nawiasów do wysokości zawartego w~nich wyrażenia należy znaki nawiasów poprzedzić poleceniami \texttt{\textbackslash left}, \texttt{\textbackslash right}\footnote{\red Tak jak to zrobiono w~pierwszym elemencie komentowanego równania. Jeśli jeden z~nawiasów ma być pominięty należy w~jego miejsce dać znak kropki.}. }

\noindent
[\ldots]

