\chapter{Podsumowanie}

Celem pracy było zapoznanie z opisem dynamiki ruchu bąków ciężkich oraz przygotowanie systemu symulacji, pozwalającego na badanie zachowania układu w~czasie, co zostało zrealizowane. W pracy przytoczono podstawowy aparat matematyczny niezbędny do\ldots{} Podano w niej opis ruchu ciała sztywnego\ldots{} Pokazano równoznaczność opisu\ldots{} Dla kompletności krótko scharakteryzowano\ldots

Przedstawiony w pracy ogólny opis dynamiki ruchu bąka został uzyskany z wykorzystaniem\ldots{} Opis zawarty w pracy umożliwia porównanie\ldots{} Matematyczne modele zostały poparte ich ilustracjami oraz, o ile było to możliwe, przykładem fizycznym.

W pracy kolejno przeprowadzono analizę jakościową równań\ldots{} W celu uzupełnienia opisu analitycznego badaniami symulacyjnymi opracowano program pozwalający na\ldots{}

{\red
  W~podsumowaniu należy przede wszystkim napisać, co było celem pracy i~w~jaki sposób został on zrealizowany. A~dalej, co dokładnie w~jej ramach zostało wykonane, jak zostało to przedstawione, na co przedstawiony materiał pozwala, co z~niego wynika. Podsumowanie powinno także zasadniczo zawierać wyodrębnioną specyfikację oryginalnego wkładu autora do pracy.}

Pomimo swojej prostoty bądź nawet prymitywności bąk potrafi zaciekawić, a~nawet zainspirować rozmaitością ruchów oraz ich ewolucji. W trakcie przeglądu literatury nie napotkano na obiekt będący uogólnieniem bąka, podobnym do uogólnienia, jakim jest podwójne wahadło dla wahadła, które może stanowić źródło interesujących zachowań, również z punktu widzenia teorii układów chaotycznych. Bąki pełnią nie tylko rolę edukacyjną, jako wprowadzenie w arkana mechaniki analitycznej, ale również rozrywkową, a nawet estetyczną. Mamy nadzieję, że przybliżenie czytelnikowi tematyki bąków zaowocuje zaopatrzeniem się w jednego z nich, puszczeniem go i medytacją nad jedną z wielu jego twarzy.

{\red
  Dalsza część podsumowania powinna zawierać wnioski płynące z~pracy, a~także wskazywać potencjalne kierunki dalszych prac\footnote{\red Czego akurat w~tej przykładowej pracy nie uczyniono.}.}
