\chapter[Specyfikacja realizacji sonaru ultradźwiękowego]{Specyfikacja realizacji sonaru ultradźwiękowego}

\label{chapter:specyfikacja}

Urządzenie oprócz dostarczania swoich podstawowych funkcji niezbędnych do działania, 
może również zaoferować pewne udogodnienia w testowaniu oraz obsłudze przez użytkownika końcowego.
Takim udogodnieniem jest na pewno zmiana istotnych parametrów sonaru poprzez komunikację szeregową z urządzeniem.
\vspace{2cm}

Komendy które przyjmuje urządzenie to:
\begin{itemize}
    \item Uruchomienie pomiaru -- rozpoczyna kompletną sekwencje pomiarową i zwraca wynik z powrotem do urządzenia.
    \item Zmiana ilości impulsów nadajnika -- za parametr przyjmuje wartości od 1 do 10 powtórzeń.
    \item Zmiana wypełnienia impulsu -- wpływa na moc nadajnika, za parametr przyjmuje wartości (0-199), które odpowiadają największemu i najmniejszemu wypełnieniu.
    \item Zmiana progu wykrywania sygnału  -- pozwala na regulacje czułości odbiornika, za parametr przyjmuje 12 bitową wartość (0-4095) przetwornika DAC.
    \item Zmiana czasu odstępu od zakończenia nadawania do rozpoczęcia odbierania, za parametr przyjmuje 
    \item Zmiana czasu końca pomiaru -- stanowi o tym kiedy mikrokontroler powinien przerwać odbieranie sygnału, za parametr przyjmuje czas wyrażony w taktach procesora o częstotliwości \unit[80]{MHz} w zakresie liczby 32 bitowej. 
\end{itemize}


% W tej części trzeba podać jakie będą udostępniane funkcjonalności,
% jak mają być realizowane pomiary, jakie polecenia będzie można
% przesyłać do urządzenia, przewidywane parametry, np.
% częstość powtórzeń pomiarów, zakres zmiany ilości sygnałów pobudzenia,
% zakres zmian wypełnienia impulsów itp.

