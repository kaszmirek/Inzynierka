\chapter{Cel i założenia}\label{ch_02}

Popularne dalmierze ultradźwiękowe wykorzystują przetworniki piezoelektryczne
jako nadajniki i odbiorniki. Ich średnice wahają się w granicach
od 10mm do 20mm.
W przypadku prostych dalmierzy, gdy wyznaczana jest tylko odległość do
obiektu, ich rozmiar nie jest krytyczny.
Jednak w konstrukcjach sonarów ultradźwiękowych, które mają wyznaczyć
również kierunek przylotu sygnału, rozmiar ten jest istotny.
Jeżeli kierunek przylotu jest wyznaczany w oparciu o przesunięcie
fazy odbieranego sygnału, wzajemna odległość odbiorników nie powinna
przekraczać pół długości fali emitowanego sygnału.
Wykorzystywane powszechnie przetworniki ultradźwiękowe pracują z częstotliwością
40kHz. Pół długości fali akustycznej w powietrzu dla tej częstotliwości
to ok. 4,3mm. Drugim warunkiem stosowalności tego podejścia
jest to, aby odbiorniki sygnału można było modelować jako punkty materialne.
Od strony technicznej oznacza to, że apertury tych odbiorników powinny być
możliwe małe w stosunku do długości fali.
Kryteriów tych nie spełniają popularne odbiorniki piezoelektryczne.

Celem niniejszej pracy jest konstrukcja sonaru pozwalającego wyznaczyć
odległość do miejsca odbicia sygnału oraz kierunku nadejścia sygnału.
Pozwalać ma to tym samym na precyzyjną lokalizację obiektu.
Zakłada się, że źródłem sygnału będzie przetwornik piezoelektryczny
pracujący z częstotliwością 40kHz.
Wyznaczanie kierunku przylotu ma zostać zrealizowane w oparciu
o przesunięcie fazy odbieranego sygnału. Chcąc spełnić opisane powyżej
warunki, jako odbiorniki zostaną zastosowane 3 mikrofony analogowe
produkowane w technologii MEMS.
Sonar powinien udostępniać komunikację poprzez interfejs USB.
Dostępna powinna być też możliwość konfiguracji jego pracy,
tzn. ilość pobudzeń generujących emitowany sygnał oraz czas opóźnienia
przejścia w tryb odbioru.
W ramach niniejsze pracy należy też zrealizować podstawowe oprogramowanie
dla komputera typu PC, które pozwoli sterować sonarem, wykonać
niezbędne pomiary oraz obliczenia. Dysponując tym oprogramowaniem
należy przeprowadzić serię eksperymentów, które pozwolą zbadać i zweryfikować
podstawowe własności sonaru. Ponadto oprogramowanie mikrokontrolera
należy zdokumentować w systemie {\tt doxygen}.
