
\section*{Streszczenie}\label{chapter:streszczenie}
Celem pracy jest budowa sonaru będącego czujnikiem ultradźwiękowym z mikrofonami MEMS. 
Urządzenie to ma posłużyć do wyznaczania kąta azymutu oraz elewacji badanego obiektu.
Zakres prac sprzętowych obejmuje projekt i wykonanie układów elektronicznych oraz obudowy urządzenia. 
Część oprogramowania wymaga opracowania programu mikrokontrolera odpowiadającego za sterowanie przetwornikami, przechwytywanie sygnałów oraz komunikację z komputerem.

Pracę podzielono na rozdziały przedstawiające następujące zagadnienia. W rozdziale drugim opisany został cel projektu oraz wymagania stawiane wobec urządzenia. 
Trzeci rozdział przedstawia proces doboru odbiorników sygnału, które są najważniejszym elementem konstrukcji. 
W czwartym pochylono się nad analizą najistotniejszych problemów występujących w tego typu sonarach. Rozdział piąty opisuje specyfikację pracy z urządzeniem wod strony użytkownika.
W rozdziale szóstym dogłębnie został przedstawiony proces projektowania układów elektronicznych oraz oprogramowania. Siódmy rozdział przedstawia realizację w postaci wizualizacji oraz opisu wykonanego urządzenia.
Przedostatni rozdział ma na celu zobrazowanie metodologii badawczej oraz efektów działania składowych elementów całego systemu. Rozdział ostatni podsumowuje pracę dyplomową.

Cel pracy został osiągnięty częściowo. Urządzenie zostało wykonane i wszystkie jego elementy przetestowane. Sprzętowa część spełnia wszystkie cele, 
lecz w obecnym systemie oprogramowania, pomimo spełnienia warunków komunikacji z użytkownikiem, nie udało się dokonać zapisu danych.

\textbf{Słowa kluczowe: }sonar, ultradźwięki, przetwornik, czujnik, systemy wbudowane, elektronika, filtr, wzmacniacz, mikrokontroler


\section*{Abstract}\label{chapter:abstract}
Goal of thesis is to make sonar device as ultrasonic sensor with MEMS microphones.
Device is going to be used to estimate angle of azimuth and elevation of tested object.
Scope of hardware work includes designing and manufacturing electronics and housing of the device. 
Software part includes developing program responsible for control of the transducers, capturing signals and communication with PC.

Thesis is divided by chapter that explains the following issues. Second chapter shows goals and assumptions of whole project. 
Third chapter explains process behind choosing the right sensor. 
Fourth chapter analyses problems that encounters in designing sonars.
Fifth chapter describes user requirements.
Sixth chapter shows process of designing hardware and software.
Seventh chapter describes implementation in form of photos and description.
Eight chapter shows test methodology and effects of working device.
Last chapter is the summary of whole thesis.

Goal of work is accomplished partly. Device has been made and all of its components were tested. 
Hardware part meets all of the expectations but in current software architecture it is impossible to save captured data fast enough.

\textbf{Keywords: }sonar, ultrasonic, transducer, sensor, embedded system, electronics, filter, amplifier, microcontroller 