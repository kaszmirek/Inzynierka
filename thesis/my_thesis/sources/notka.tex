%\newpage
%\thispagestyle{empty}
%\cleardoublepage
%\thispagestyle{plain}

\mbox{}\vfill\hfill
\begin{minipage}{0.5\linewidth} 
  {\tiny \noindent Do składu pracy wykorzystano system przygotowania
    dokumentów~\LaTeX, opracowany przez
    L.~Lamporta\index{latex>\LaTeX} [Lam94], będący nakładką
    systemu \TeX, [Knu86a,Knu86b].  Matematyczne czcionki o nazwie
    {AMS Euler}, których używamy w tej pracy, zostały przygotowane
    przez H.\ Zapfa [KZ86], przy współpracy z~D.\ Knuthem i~jego
    studentami, na zlecenie Amerykańskiego Towarzystwa Matematycznego.
    %% Przy wybranej Antykwie Toruńskiej/Półtawskiego odznacz odpowiednio poniższe
    Wybrane czcionki składu tekstu, Antykwa Toruńska [Now97] -- jeden
    %Wybrane czcionki składu tekstu, Antykwa Półtawskiego [Now99] -- jeden
    z~nielicznych krojów pisma zaprojektowany specjalnie dla języka
    polskiego w~sposób uwzględniający jego rytm -- w~odczuciu autora
    doskonale współgrają z~kształtem czcionki {AMS Euler}, pozwalając
    na uzyskanie harmonijnej całości.
    % %% Przy wybranych czcionkach Concrete odznacz poniższe
    % Czcionki składu tekstu, zwane {Concrete Roman} i {Concrete
    %   Italic}, należące do knuthowskiej rodziny czcionek {Computer
    %   Modern}, zostały specjalnie przystosowane do kształtu czcionki
    % {AMS Euler} na potrzeby książki [GKP96].
    % %% Przy wybranych czcionkach URW Palladio odznacz poniższe
    % Czcionka składu tekstu, zwana URW Palladio jest klonem zapfoskiej rodziny
    % czcionek o~nazwie Palatino [LPn05] i~zdaniem autora świetnie współgra
    % z~kształtem czcionki {AMS Euler}.
    Składu bezszeryfowego tekstu maszynowego dokonano z~użyciem
    opracowanej przez R. Leviena czcionki o~nazwie Inconsolata
    [Lev15]\footnote{\red\tiny Chyba warto takie informacje szerzyć}.


\vspace{-4mm}

 \makeatletter
\renewenvironment{thebibliography}[1]
     {%
        \tiny%
      \list{\@biblabel{\@arabic\c@enumiv}}%
           {\settowidth\labelwidth{\@biblabel{#1}}%
\setlength{\itemsep}{2.5mm}
            \leftmargin\labelwidth
            \advance\leftmargin\labelsep
            \@openbib@code
            \usecounter{enumiv}%
            \let\p@enumiv\@empty
            \renewcommand\theenumiv{\@arabic\c@enumiv}}%
      \sloppy\clubpenalty4000\widowpenalty4000%
      \sfcode`\.\@m\vspace{5mm}}
     {\def\@noitemerr
       {\@latex@warning{Empty `thebibliography' environment}}%
      \endlist}

\makeatother

\begin{thebibliography}{Knu86b}

% %% Odmarkować pozycję gdy wybrane czcionki Concrete
% \bibitem[GKP96]{GKP96loc}
% R.~L. Graham, D.~E. Knuth i O.~Patashnik,
% \newblock { Matematyka konkretna}.
% \newblock PWN, Warszawa, 1996.\vspace{-3mm}

\bibitem[Knu86a]{Knuth86loc}
D.~E. Knuth,
\newblock { The \TeX book, volume {A} of Computers and Typesetting}.
\newblock Addison-Wesley, Reading, 1986.\vspace{-3mm}

\bibitem[Knu86b]{Knuth86aloc}
D.~E. Knuth,
\newblock { \TeX: {The} Program, volume {B} of Computers and Typesetting}.
\newblock Addison-Wesley, Reading, 1986.\vspace{-3mm}

\bibitem[KZ86]{KnZa89loc}
D.~E. Knuth i H.~Zapf,
\newblock {AMS} {Euler} --- {A} new typeface for mathematics.
\newblock { Scholary Publishing}, {20}:131--157, 1986.\vspace{-3mm}

\bibitem[Lam94]{Lamport94loc}
L.~Lamport,
\newblock { \LaTeX: A Document Preparation System}.
\newblock Addison-\mbox{-Wesley}, Reading, 1994.\vspace{-3mm}

\bibitem[Lev15]{Levien15loc}
R.~Levien,
\newblock {Inconsolata}.
\newblock \url{https://levien.com/type/myfonts/inconsolata.html}, 2015.\vspace{-3mm}

% %% Odmarkować pozycję przy wybranej czcionce URW Palladio
% \bibitem[LPn05]{LinotypePalatino05loc}
% Linotype Palatino nova: A classical typeface redesigned by Hermann Zapf,
% \newblock Linotype Library GmbH, 2005.\vspace{-2mm}

%% Odmarkować pozycję przy wybranej Antykwie Toruńskiej
\bibitem[Now97]{nowacki97loc}
J.~Nowacki,
\newblock {Antykwa} {Toruńska} -– od początku do końca polska czcionka.
\newblock {\em Biuletyn Polskiej Grupy Użytkowników Systemu \TeX}, 9:26--27,
  \nolinebreak1997.\vspace{-2mm}

% %% Odmarkować pozycję przy wybranej Antykwie Półtawskiego
% \bibitem[Now99]{nowacki99}
% J.~Nowacki,
% \newblock Piórkiem i {MetaPost-em}, czyli {Antykwa} {Półtawskiego}.
% \newblock {\em Biuletyn Polskiej Grupy Użytkowników Systemu \TeX}, 12:49--53,
%   \nolinebreak1999.\vspace{-2mm}

\end{thebibliography}
}
\end{minipage}
