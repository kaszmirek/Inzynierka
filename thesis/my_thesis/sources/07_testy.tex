\chapter[Testy i eksperymenty]{Testy i eksperymenty}

\label{testy}

Opis zrealizowanych eksperymentów, które demonstrują
najważniejsze cechy urządzenia i czujnika.

\section{Test przetwornika piezoelektrycznego}

\begin{figure}[!ht]
    \centering
    \missingfigure{screen z oscylo z przebiegiem sygnału z piezo}
    \caption{Przebieg sygnału odebrany innym przetwornikiem piezoelektrycznym}
    \label{fig:oscylo_piezo}
\end{figure}

\section{Test wpływu odległości na sygnał}
Z identycznym stanowiskiem pomiarowym co sekcję wyżej sprawdzono wpływ odległości czujników na moc i przesuniecie fazy sygnału. 
\todo{zdjęcie z oscylo z przesuniętym sygnałem i kilka testów na różne odległości}

\begin{figure}[!ht]
    \centering
    \missingfigure{screen z oscylo z przebiegiem sygnału z piezo}
    \caption{Przebieg sygnału odebrany innym przetwornikiem piezoelektrycznym, wpływ na odległość}
    \label{fig:oscylo_2piezo}
\end{figure}

\section{Pierwsze uruchomeinie}
Pierwsze uruchomienie elektroniki ujawniło drobny błąd projektowy, wszystkie LEDY sygnalizacyjne zostały przylutowane w złej polaryzacji.
Szybka zmiana ustawień diod i następne uruchomienie, nie pokazywało oznak większych błędów. Pobór prądu\todo{podać ile prądu ciągnie}, jak i temperatura elementów na 
płytce była w normie.

\section{Uruchomienie i test wzmacniacza sygnału przetwornika piezoelektrycznego}

\begin{figure}[!ht]
    \centering
    \missingfigure{screen z oscylo z przebiegiem sygnału z piezo}
    \caption{Przebieg sygnału odebrany innym przetwornikiem piezoelektrycznym}
    \label{fig:oscylo_piezo}
\end{figure}

\section{Test mikrofonów i filtrów}


\begin{figure}[!ht]
    \centering
    \missingfigure{screen z oscylo z przebiegiem sygnału z piezo}
    \caption{Przebieg sygnału odebrany innym przetwornikiem piezoelektrycznym}
    \label{fig:oscylo_piezo}
\end{figure}

