\chapter{Wstęp}
\markboth{Wstęp}{Wstęp}
\label{wstep}


    % Celem rozdziału jest zaprezentowanie podstawowych informacji dotyczących dalmierzy ultradźwiękowych, zasady działania i~ich zastosowania w przemyśle, nauce oraz życiu codziennym. 
    % Ma on również za zadanie przybliżyć rozwinięcia skrótów powszechnie używanych w tej dziedzinie.

Wiele zaawansowanych współczesnych technologii czerpie pomysły bezpośrednio z natury. Jednym z takich rozwiązań jest echolokacja, 
wykorzystywania przez zwierzęta takie jak nietoperze czy delfiny, do nawigacji w przestrzeni. 
Umiejętność ta wytworzyła się naturalnie w drodze ewolucji.

Według Wikipedii \cite{sonar},
sonar to pojęcie obejmujące szeroką grupę urządzeń służących do komunikacji, nawigacji, detekcji i klasyfikacji obiektów znajdujących się w cieczy bądź powietrzu.
Nazwa ta pochodzi z języka angielskiego i jest akronimem od wyrażenia {\em ,,\textbf{SO}und \textbf{N}avigation \textbf{A}nd \textbf{R}anging", 
co tłumaczy się na ,,nawigacja dźwiękowa i pomiar odległości".}
Urządzenia mogą wykorzystywać szeroki zakres fal dźwiękowych jako nośnika informacji, od infradźwięków, aż po ultradźwięki.

Podstawą działania sonarów jest efekt propagacji fali dźwiękowej miedzy nadajnikiem a odbiornikiem. 
Sonary aktywne są wyposażone w nadajnik transmitujący falę oraz odbiornik, który ją odbiera. 
Mierzone są zatem parametry sygnału odbitego od wykrywanego obiektu, takie jak czas powrotu fali 
czy, w przypadku bardziej zaawansowanych konstrukcji, również kierunek, z którego nadchodzi fala.

Obecnie sonary wykorzystywane są w bardzo wielu dziedzinach nauki, przemyśle, medycynie, wojsku a także w życiu codziennym. 
Sonary bardzo dobrze radzą sobie w wodzie, ze względu na większą prędkość dźwięku w gęstszych ośrodkach. 
Najbardziej znanym zastosowaniem tych urządzeń jest aparatura pomiarowa będąca na wyposażeniu łodzi podwodnych oraz rybołówstwo. 
Pozwalają one na precyzyjne określenie kształtu den mórz oraz rozmiary ławic. W łodziach podwodnych system ten wykorzystywany jest głównie do wykrywania obiektów kolizyjnych.
Ze względu na błyskawiczny rozwój przemysłu czujniki ultradźwiękowe mają coraz to szersze zastosowanie, 
między innymi do automatyzacji procesów linii produkcyjnych oraz wykrywania kolizji autonomicznych robotów mobilnych. 
Świetnie się sprawdzają również w analizie zmęczeniowej materiałów, co pozwala na wczesne reagowanie w przypadku awarii kluczowych elementów dużych konstrukcji.
Na co dzień rozwiązanie to spotkać można chociażby w samochodach wyposażonych w czujniki parkowania.

Wspomniane zastosowania to zaledwie ułamek faktycznego wykorzystania tej technologii. 
Pomimo stu lat rozwoju sonarów, w dziedzinie tej pozostaje wiele do odkrycia. 
Najbliższe lata mogą przynieść coraz to nowsze, kreatywne metody użycia ich w nieznanych dotąd przykładach.

\nocite{sonar}